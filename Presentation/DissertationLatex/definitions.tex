%%
%% Place here your \usepackage's. Some recommended packages are already included.
%%

% Graphics:
\usepackage[final]{graphicx}
%\usepackage{graphicx} % use this line instead of the above to suppress graphics in draft copies
%\usepackage{graphpap} % \defines the \graphpaper command

% Indent first line of each section:
\usepackage{indentfirst}

% Good AMS stuff:
\usepackage{amsthm} % facilities for theorem-like environments
\usepackage[tbtags]{amsmath} % a lot of good stuff!

% Fonts and symbols:
\usepackage{amsfonts}
\usepackage{amssymb}

\usepackage{xspace}
\usepackage{algorithmic}
\usepackage{algorithm}
\usepackage{microtype}
\usepackage{subfigure}
\usepackage{color}
\usepackage{todonotes}
\usepackage{url}
\newfloat{algorithm}{t}{lop}

% Formatting tools:
%\usepackage{relsize} % relative font size selection, provides commands \textsmalle, \textlarger
%\usepackage{xspace} % gentle spacing in macros, such as \newcommand{\acims}{\textsc{acim}s\xspace}

% Page formatting utility:
\usepackage{geometry}
\usepackage{multirow}
\usepackage{listings}
%
\usepackage[round,authoryear]{natbib}
\usepackage[all,cmtip]{xy}
\usepackage[colorlinks=true,citecolor=blue,urlcolor=blue, filecolor=blue,pdfpagemode=UseNone,pdfstartview=FitH, final]{hyperref}
%\usepackage[breaklinks=true,a4paper=true,pagebackref=true]{hyperref}
\usepackage{tocloft}
\newcommand{\Color}[1]{\hypersetup{linkcolor=#1}\color{#1}}
\renewcommand{\cftchapfont}{\Color{black}\large\bfseries}
\renewcommand{\cftsecfont}{\Color{black}}
\renewcommand{\cftsubsecfont}{\Color{black}}
\renewcommand{\cftsubsubsecfont}{\Color{black}}
\renewcommand{\cftchappagefont}{\Color{black}}
\renewcommand{\cftsecpagefont}{\Color{black}}
\renewcommand{\cftsubsecpagefont}{\Color{black}}
\renewcommand{\cftsubsubsecpagefont}{\Color{black}}
\renewcommand{\cftsecleader}{\Color{black}\cftdotfill{\cftsecdotsep}}
\renewcommand{\cftsubsecleader}{\Color{black}\cftdotfill{\cftsubsecdotsep}}
\renewcommand{\cftsubsubsecleader}{\Color{black}\cftdotfill{\cftsubsubsecdotsep}}
\renewcommand{\cftchapleader}{\large\bfseries\color{black}\cftdotfill{\cftdotsep}}
\renewcommand{\cftchappagefont}{\large\bfseries}
%\usepackage{hyperref}
%\hypersetup{
 %   colorlinks=true,
  %  linkcolor=blue,
   % filecolor=blue,
%    urlcolor=blue,
 %    citecolor=blue
 % }

%%
%% Place here your \newcommand's and \renewcommand's. Some examples already included.
%%
%\newcommand{\acims}{\textsc{acim}s\xspace}
\newcommand{\Mspace}        {{\mathbb M}}
\newcommand{\Rspace}        {{\mathbb R}}
\newcommand{\Cspace}        {{\mathbb C}}

\newcommand{\Mo}        {{\hat M}}
\newcommand{\Ms}        {{\tilde M}}
\newcommand{\Do}          {{\hat D}}
\newcommand{\Ds}        {{\tilde D}}
\newcommand{\doo}          {{\hat d}}
\newcommand{\dss}        {{\tilde d}}
\newcommand{\w}        {{\mathbf w}}

% general
\newcommand{\ie}{i.e.}
\newcommand{\eg}{e.g.}
\newcommand{\reffig}[1]{{Figure~\ref{#1}}}
\newcommand{\refchap}[1]{{Chapter~\ref{#1}}}
\newcommand{\refsec}[1]{{Section~\ref{#1}}}
\newcommand{\reftab}[1]{{Table~\ref{#1}}}
\newcommand{\refapp}[1]{{Appendix~\ref{#1}}}
\newcommand{\refeq}[1]{{Equation~\ref{#1}}}
\newcommand{\refalg}[1]{{Algorithm~\ref{#1}}}
\newcommand{\myparagraph}[1]{\noindent \textbf{#1}}
\newcommand{\highlight}[1]{{\color{black}#1}}


%%%%RAFA PACKAGES
\usepackage[font=bf, justification=centering]{caption}
\DeclareUnicodeCharacter{00A0}{'}
\usepackage{longtable}    
\usepackage{lscape}
\usepackage{booktabs}
%\usepackage[colorlinks=true,citecolor=blue,urlcolor=blue,pdfpagemode=UseNone,pdfstartview=FitH]{hyperref}

%%
%% Place here your \newtheorem's:
%%

%% Some examples commented out below. Create your own or use these...
%%%%%%%%%\swapnumbers % this makes the numbers appear before the statement name.
%\theoremstyle{plain}
%\newtheorem{thm}{Theorem}[chapter]
%\newtheorem{prop}[thm]{Proposition}
%\newtheorem{lemma}[thm]{Lemma}
%\newtheorem{cor}[thm]{Corollary}

%\theoremstyle{definition}
%\newtheorem{define}{Definition}[chapter]

%\theoremstyle{remark}
%\newtheorem*{rmk*}{Remark}
%\newtheorem*{rmks*}{Remarks}

%% This defines the "proo" environment, which is the same as proof, but
%% with "Proof:" instead of "Proof.". I prefer the former.
%\newenvironment{proo}{\begin{proof}[Proof:]}{\end{proof}}
