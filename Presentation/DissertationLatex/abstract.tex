%!TEX root = thesis.tex

This dissertation studies what prevents (or fosters) would-be contemporary nation-builders from building the nation and solidifying the state when challenged by criminal wars. Focusing on Mexico and Colombia -two states rampaged by ongoing criminal wars-, I study the incentives of local governments electoral incentives (Chapter 1 and 2), and the way in which non-state armed groups capture them and transform local institutions to their preferences (Chapter 3). 

The first chapter titled ``Reelection Backfire: How Reelection Concerns Affect the Delegation of Public Security Policy in Mexico'' studies the effect of reelection incentives of mayors on the delegation of security policy to governors in Mexico. It is no easy task for local governments to combat organized crime and hold the monopoly of violence. Faced by resource, expertise, or information constraints, or in the presence of large spillovers, mayors may choose to delegate security policy to more efficient agents like the governor or the president. However, by doing so mayors loose the use of security policy for electoral purposes: they cannot longer claim competence and willingness to fight crime or be responsive to citizens demands. A clear tradeoff between efficiency and electoral incentives arises. This paper studies the effect of mayors' reelection incentives on the delegation of security policy to the governor in a country overwhelmed by criminal wars, Mexico. To do so, I leverage the staggered implementation of an electoral reform that introduced reelection for mayors from 2014 to 2022. I find that mayors up for reelection decrease the delegation of public security to the governor of their state relative to term-limited mayors. By taking charge of policy, mayors facing reelection differentiate themselves from other political actors believing this would increase their reelection chances. However, no evidence is found that mayors increase their effort to fight crime and address citizens security concerns. The consequence of not delegating to more efficient actors, however, is an increase in violence. This paper suggests that delegation is not only a policy decision but an electoral one, and that reelection incentives may lead incumbents to differentiate themselves by taking charge of public policy at the expense of efficiency.   

The second chapter titled ``Love the Candidate but Hate his Party: The Asymmetric Effects of Reelection Incentives on Partisan and Personal Incumbency Returns in Mexico'' studies incumbency returns to office in local governments. A large literature has studied the electoral returns to incumbency. However, the estimated incumbency returns in the majority of studies confound the returns to the incumbent party and politician. As a result we may hold a misguided picture of the relationship between parties and their members, citizens valuation of the electoral system, and the existent electoral accountability. This paper opens up the black box of incumbency by disentangling the personal and partisan incumbency advantages. To do so, I use a difference-in-discontinuity of close elections design in Mexico that exploits the staggered implementation of an electoral reform that introduced reelection for mayors from 2014 to 2022. Term limit elections allow us to identify a partisan advantage since the politician cannot run for office again but his party can. Elections with candidates up for reelection identify both the partisan and personal incumbency returns.  The difference between this two quantities -or the difference-in-discontinuity estimator- identifies the personal from the partisan effect. The main result shows that incumbent politicians up for reelection enjoyed and electoral \emph{advantage}. Contrastingly, incumbent parties in term limit races suffer from an incumbency \emph{disadvantage}. The research design allows to solve several methodological issues of past studies that have tried to disentangle the partisan from the personal effect, primarily rule out potential pretrends of term limit and non-term limit races, as well as concerns on selection coming from the ability and experience of candidates. Overall, the results suggest the personal advantage to be a driving force of incumbency returns, and how  reelection may lead to party dealignment in party-centered systems like Mexico. 

Lastly, the third chapter titled ``Endogenous taxation in ongoing internal conflict: The case of Colombia'' studies how non-state armed groups capture municipalities and transform local institutions according to their preferences. Recent empirical evidence suggests an ambiguous relationship between internal conflicts, state capacity and tax performance. In theory, internal conflict should create strong incentives for governments to develop the fiscal capacity necessary to defeat rivals. In this chapter we argue that one reason that this does not occur is because internal conflict enables groups with {\it{de facto}} power to capture local fiscal and property rights institutions. We test this mechanism in Colombia using data on tax performance and property rights institutions at the municipal level. Municipalities affected by internal conflict have tax institutions consistent with the preferences of the parties dominating local violence. Those suffering more right-wing violence feature more land formalization and higher property tax revenues. Municipalities with substantial left-wing guerrilla violence collect less tax revenue and witness less land formalization. Our findings provide systematic evidence that internal armed conflict helps interest groups capture municipal institutions for their own private benefit, impeding state-building.

