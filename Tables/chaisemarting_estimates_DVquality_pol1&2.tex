\begin{table}[htbp]\def\sym#1{\ifmmode^{#1}\else\(^{#1}\)\fi}
\centering
\caption{Effect of Term Limit Reform on Incumbency Advantage}
\label{tab:chaisemartin}
\scalebox{0.65}{
\begin{tabular}{lcc}
\hline \hline
\\ \multicolumn{3}{l}{Dependent variable: incumbent party at t-1 won at t dummy}\\
& \multicolumn{1}{c}{linear polynomial} & \multicolumn{1}{c}{quadratic polynomial} \\
& \multicolumn{1}{c}{(1)} & \multicolumn{1}{c}{(2)} \\
\cmidrule(lrr){2-2}  \cmidrule(lrr){3-3}\\
\addlinespace
3 elections before &        $ -0.046^{*} $ &     $ -0.041^{} $ \\
& ($ 0.026$) & ($ 0.026 $) \\
2 elections before &        $ -0.012^{} $ &     $ -0.016^{} $ \\
& ($ 0.015$) & ($ 0.019 $) \\
t=0 reform &        $ 0.003^{} $ &     $ -0.009^{} $ \\
& ($ 0.039$) & ($ 0.033 $) \\
1 election after &         $ 0.073^{***} $ &       $ 0.062^{**} $ \\
& ($ 0.010$) & ($ 0.030 $) \\
\addlinespace
Controls   &    \checkmark      &   \checkmark    \\
\hline \hline
\multicolumn{3}{p{0.8\textwidth}}{\footnotesize{Notes: State clustered standard errors in parentheses are clustered at the state level, with the following significance-level: $^{***}$ 1\%; $^{**}$ 5\%; and $^*$ 10\%. Forcing variable=winning margin. Optimal bandwidth from \textcolor{blue}{Calonico et al. 2014}.}} \\
\end{tabular}
}
\end{table}
